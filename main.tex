\documentclass[12pt]{article}
\usepackage[spanish]{babel}
\usepackage{amsmath}
\usepackage{graphicx}

\begin{document}

\begin{center}
\bf{\sc\Huge universidad de antioquia}\\
\end{center}
\vspace{120pt}
\begin{center}
\bf{\sc\Huge sara valentina agudelo  }\\
\end{center}
\vspace{200pt}
\begin{center}
\bf{\sc\Huge medellín}
\end{center}
\begin{center}
\bf{\sc\Huge 2020}\\
\end{center}\
\newpage



\begin{center}

\bf{\sc\Huge La historia de la computación }\\
\end{center}
\begin{flushleft}
\vspace{25PT}
\large
""Las máquinas me sorprenden con mucha frecuencia" 


-Alan Turing
\end{flushleft}
\vspace{}

\large

A lo largo de la historia de la humanidad, el hombre siempre ha buscado la forma de avanzar, de facilitar el modo de hacer de las cosas, porque esa es nuestra naturaleza, desde que el hombre comenzó con la caza y la cosecha hasta hoy se puede evidenciar este comportamiento. 


\vspace{}
\vspace{30pt}
La definición de una máquina es “objeto fabricado y compuesto por un conjunto de piezas ajustadas entre sí que se usa para facilitar o realizar un trabajo determinado”, así que podríamos llamar máquina a todo lo que sirva o ha servido al ser humano para realizar una tarea fácilmente. 

\vspace{30pt}
Las primeras máquinas de la historia aparecieron en la prehistoria, cuando los primeros hombres quisieron facilitar su modo de caza y recolección, la forma de contar las cosas se fue haciendo cada vez más importante y en 1642 Blaise Pascal creó la que sería la primera calculadora: el ábaco.

\vspace{15pt}
Con la llegada de la revolución industrial, la tecnología y la ciencia, las matemáticas tomaron un papel muy importante en el cambio de la sociedad, en ese momento la precisión de las matemáticas era algo fundamental, pero a finales de siglo surgieron algunas preguntas que hicieron cuestionarse a diferentes personas que conocían el campo de las matemáticas si estas eran tan precisas como parecían, pues las matemáticas siempre fueron consideradas como algo absoluto , los enunciados y las proposiciones debían ser verdaderas o falsas, blanco o negro  y una de esas preguntas que hicieron que esta consideración cambiara  fue el infinito. 

\vspace{15PT}
El infinito ha sido un concepto que a través de la historia ha causado confusión, los griegos intentaron comprenderlo, pero basándose en lo finito, por esta razón empezaron a surgir cosas que eran contrarias a la lógica, las llamadas paradojas.

\vspace{15PT}
Hubo un momento de la historia en que la comunidad científica estaba dividida en dos bandos, uno de ellos llamado:   “los formalistas”, los formalistas aseguraban que todo se podía probar a través de la matemáticas e incluso con un estudio riguroso se podía evitar las paradojas que surgían , una de estas personas fue David Hilbert, el afirmaba que si se estudiaban bien los axiomas matemáticos se podría evitar cualquier inconveniente que se generara al demostrar algo como el infinito sin ninguna paradoja .
 
\vspace{15PT}
Algunas otras personas como George Cantor, dedicaron gran parte de su vida al estudio de estas paradojas, Cantor fue la primera persona en “definir” el infinito y como se comportaba.

\vspace{15PT}
Cantor, se dio cuenta que existían ciertas paradojas que hicieron que esa afirmación de que las matemáticas eran infalibles se tambaleara, así comenzó lo que se llamaría “la crisis de los fundamentos”, 
Otras dos personas que tuvieron participación en esta parte de la historia fueron Kurt Godel y Alan Turing. Godel logro probar que en este universo matemático había cosas que no se podían demostrar.

\vspace{15PT}

Alan Turing diseñó un dispositivo hipotético al que más adelante se le llamó “máquina de Turing “, este dispositivo manipulaba unos símbolos sobre una cinta y de acuerdo a unas reglas determinadas era posible hacer cálculos matemáticos con ella, de esta forma se podría realizar cualquier cálculo mecánico, de hecho, gracias a Alan Turing nacieron las primeras máquinas, se dice que Turing es el “padre de la computación “.

\vspace{15PT}
 Pero había un problema, si la máquina decía que un problema tenía una solución, procedía a buscar la solución, pero en realidad no se sabía en qué momento de tiempo se iba a encontrar esta solución, podrían ser días, meses e incluso años, por lo tanto, no todos los problemas se pueden resolver por el método matemático, por simple lógica, a esto se le llamó un problema “indecidible”. 

\vspace{15PT}
Como resultado de todas estas demostraciones, se logra evidenciar que es algo cierto que no todo se puede medir con las matemáticas, no en este universo, pero hay algunas corrientes científicas que afirman que en realidad si podríamos saber con certeza cuando un resultado se va a dar, podría haber una forma de que todo en el universo se pudiera explicar de manera exacta, por ejemplo, en el mundo cuántico.

\vspace{15PT}
Gracias al avance de la tecnología podríamos llegar a un momento en el que podamos medir todo con certeza, por ahora apenas está surgiendo lo que se conoce como la computación cuántica y con ello la humanidad se está abriendo caminos a lo que sería una revolución de como conocemos las cosas, pues nos hemos dado cuenta que no todo se puede explicar desde el mundo físico, por ahora. 

\vspace{15pt}


FUENTES:


\vspace{15pt}
\vspace{15pt}
ELÍAS BARO GONZALES Y AMADOR MARTÍN 
\vspace{15pt}
25 de enero de 2019
https://www.investigacionyciencia.es/revistas/investigacion-y-ciencia/formacin-de-nuevos-rganos-144/un-alan-turing-desconocido-7473
https://elpais.com/elpais/2019/01/24/ciencia/1548329597_971134.html

\vspace{15pt}

JUAN APARICIO BELMONTE

23 de septiembre de 2019
https://elcultural.com/el-infinito

\vspace{15pt}
6 de octubre de 2016
\vspace{15pt}
https://www.nationalgeographic.com.es/ciencia/alan-turing-pensando-en-maquinas-que-piensan_9747

\vspace{15pt}
Revista Cubana de Filosofía
La Habana, enero-diciembre de 1950

MARIO O. GONZALES

\vspace{15pt}
http://www.filosofia.org/hem/dep/rcf/n06p025.htm

\vspace{15pt}
https://es.wikipedia.org/wiki/M%C3%A1quina

\end{document}
